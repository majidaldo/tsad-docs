%%% Local Variables:
%%% mode: latex
%%% TeX-master: t
%%% End:


%%
%% GMU LaTeX MS Thesis Format Template
%%
%% Developed by:
%%      Daniel O. Awduche and Christopher A. St. Jean
%%      Communications and Networking Lab
%%      Dept. of Electrical and Computer Engineering
%%
%% Notes on usage can be found in the accompanying USAGE_NOTES.txt file.
%%
%%**********************************************************************
%% Legal Notice:
%% This code is offered as-is without any warranty either
%% expressed or implied; without even the implied warranty of
%% MERCHANTABILITY or FITNESS FOR A PARTICULAR PURPOSE!
%% User assumes all risk.
%% In no event shall any contributor to this code be liable for any damages
%% or losses, including, but not limited to, incidental, consequential, or
%% any other damages, resulting from the use or misuse of any information
%% contained here.
%%**********************************************************************
%%
%% $Id: GMU_thesis_template.tex,v 1.16 2007/05/02 02:20:11 Owner Exp $
%%

\documentclass[11 pt]{report}

%%  The file ``gmuthesis.sty''  is the GMU latex style file and
%%   should be placed in the same directory as your LaTeX files
\usepackage{gmuthesis}

%%
%% other packages that need to be loaded
%%

\usepackage{graphicx}                    %   for imported graphics
\usepackage{amsmath}                     %%
\usepackage{amsfonts}                    %%  for AMS mathematics
\usepackage{amssymb}                     %%
\usepackage{amsthm}                      %%
\usepackage[normalem]{ulem}              %   a nice standard underline package
\usepackage[noadjust,verbose,sort]{cite} %   arranges reference citations neatly
\usepackage{setspace}                    %   for line spacing commands
\usepackage{float}                       %for precise placemnt of figs
\usepackage{subcaption} \usepackage{caption}
\usepackage{algorithmic} \usepackage{algorithm}
\usepackage{csvsimple}
\usepackage{hyperref}
\usepackage{enumitem}
%% The file ``mythesisabbrev.sty'' is an (optional) personalized file that
%% may contain any and all LaTeX command (re)definitions that will be used
%% throughout the document
%\usepackage{mythesisabbrev}
\usepackage{amsmath,amssymb}
\newcommand{\mt}[1]{\ensuremath{\mathbf{#1}}}
\newcommand{\vc}[1]{\ensuremath{\boldsymbol{#1}}}



% % fig stuff does not work!
% \usepackage{standalone}
% \usepackage{pgf}
% \usepackage{tikz}

\beforedoc

\begin{document}

%% In this section, all of the user-specific fields to be used in the
%% title pages are set
\title{
  Unsupervised Anomaly Detection in Sequences\\
  Using Long Short Term Memory \\
  Recurrent Neural Networks} %todo ok title?
\onelinetitle{
  Unsupervised Anomaly Detection in Sequences
  Using Long Short Term Memory 
  Recurrent Neural Networks
} %todo ok title?
\author{
Majid S. alDosari
}
\degree{Master of Science}
\subject{(Computational Science)}
\doctype{Thesis}
\dept{Computational and Data Sciences}
\degreeyear{2016}


\firstdeg{Bachelor of Science}
\firstdegschool{Vanderbilt University}
\firstdegyear{2003}
\seconddeg{Master of Science}
\seconddegschool{Vanderbilt University}
\seconddegyear{2012}


% Note: semester name should be written in its full-form. For example, Fall Semester, not just Fall.
\degreesemester{Spring Semester}

\advisor{Dr. Estella Blaisten-Barojas} %todo: middle inital?
\firstmember{Dr. Kirk D. Borne}
\secondmember{Dr. Igor Griva} 
\depthead{Dr. Kevin Curtin (acting)}
\assocdean{Dr. Donna Fox}
\dean{Dr. Peggy Agouris}


%%
%% Introductory pages
%%

% Note: The signature sheet is set according to the requirements of the Volgenau School of
% Information Technology and Engineering. If your college/school requirement is different,
% please make appropriate changes in the "signaturepage" section of gmudissertation.sty file.
\signaturepage

\titlepage

% copyright technically optional but should be included in to avoid potential pagination problems
\copyrightpage

%%
%% Dedication page
%%

\dedicationpage

\noindent I dedicate this thesis to my father, Saad F. al-Dosari, who supported me in this endeavor.

%%
%% Acknowledgements
%%

\acknowledgementspage

\noindent
I appreciate Dr. Borne's lead, as well as encouraging enthusiasm, in this endeavor.
%
Similarly, I am grateful to my other committee members, Dr. Blaisten-Barojas and Dr. Griva, who have given their time and input so that I may successfully complete my thesis.
%
Also, I give special thanks to John Kaufhold of Deep Learning Analytics for being responsive to my questions regarding anything related to neural networks.
%
His interest in and support of my work motivated me to do the best job that I can.
%
Last but not least, I appreciate very much that Leif Johnson, author of the \texttt{theanets} neural network package that I used, provided assistance beyond user support.

%%
%% Table of contents, list of tables, and lists of figures
%%

\tableofcontents

\listoftables

\listoffigures

%%
%% Abstract
%%
\abstractpage

Recurrent neural networks (RNN) with Long Short Term Memory (LSTM) are evaluated for their potential to generically detect anomalies in sequences.
%
First, anomaly detection techniques are surveyed at a high level so that their shortcomings are exposed.
%
The shortcomings are mainly their inflexibility in the use of a context `window' size and/or their suboptimal performance in handling sequences.
%
Furthermore, high-performing techniques for sequences are usually associated with their respective knowledge domains.
%
After discussing these shortcomings, RNNs are exposed mathematically as generic sequence modelers that can handle sequences of arbitrary length.
%
From there, results from experiments with RNNs show their ability to detect anomalies in a set of test sequences.
%
The test sequences had different types of anomalies but each had its unique normal behavior.
%
Given the characteristics of the test data, it was concluded that the RNNs were not only able to generically distinguish rare values in the data (out of context) but were also able to generically distinguish abnormal \emph{patterns} (in context).


%% Be sure to leave a line of whitespace immediately before this line!!!!!
%% (If this comment segment runs together with the preceeding text, you might
%%  see the second page of the abstract numbered "0".)
%%
%% If the abstract is more than one page, then place this line PRECISELY
%% at the page break; otherwise, comment it out.  (See note about this line
%% in the usage notes.)
%%
%\abstractmultiplepage

%The second page of the abstract

%%
%%  the main body of the dissertation
%%
\startofchapters

%% include the chapters one by one (or paste the chapter text in directly if desired)
\chapter[intro]{Introduction}


In the modern world, large amounts of time series \footnote{time series may also be called sequences} data of various types are recorded.  Inexpensive and compact instrumentation and storage allows various types of processes to be recorded. For example, human activity being recorded includes physiological signals, automotive traffic, website navigation activity, and communication network traffic. Other kinds of data are captured from instrumentation in industrial processes, automobiles, space probes, telescopes, geological formations, oceans, power lines, and residential thermostats. Furthermore, the data can be machine generated for diagnostic purposes such as web server logs, system startup logs, and satellite status logs.

Increasingly, these data are being analyzed. Inexpensive and ubiquitous networking has allowed the data to be transmitted for processing. At the same time, ubiquitous computing has allowed the data to be processed at the location of capture.

While the data can be recorded for historical purposes, much value can be obtained from finding anomalous data. However, it is challenging to go through large and varied quantities of data to find anomalies. Even if a procedure can be developed for one type of data, it usually cannot be applied to another type of data.

Hence, the problem that is addressed can be stated as follows: find anomalous points in an arbitrary time series. So, a solution must use the same procedure to analyze different types of time series data. In the language of machine learning, this problem is unsupervised.

A literature search presents at least two such solutions (though not much more). In the acoustics domain, \cite{Marchi2015} transform audio signals into a sequence of spectral features which are then used as input to a denoising recurrent autoencoder. \cite{Malhotra2015} improve on this by using recurrent neural networks (directly) without the use of features (that are speific to acoustics) to multiple domains.

This work closely resembles \cite{Malhotra2015} but with emphasis on automating the procedure so that it applies to many domains. But first, some background is given on anomaly detetion that explains the challenges of finding a solution. Second, recurrent neural networks are introduced as general sequence modelers. Then, experiments will be presented to show that recurrent neural networks can find anomalies in multiple domains. Concluding remarks finalize the thesis.

%todo pcc project

%%% Local Variables:
%%% mode: latex
%%% TeX-master: "thesis"
%%% End:

%% chapter was hard to write b/c
% - survery of surveys 
% - conflicting information
% - little indication of what method to use for what kind of data
\chapter[adts]{Anomaly Detection in Time Series}

\section[adintro]{Introduction}

The goal of this chapter is to show that the solution to the general problem of anomaly detection in time series is difficult. A typical general framework for anomaly detection in time series is explained with two advanced solutions as examples and their issues. The proposed solution, explained later, fits into the framework providing a basis for comparison.

Describing the variety of solutions puts this difficulty in context. Few publications survery the problem of anomaly detection for time series in particular \cite{Cheboli2010} \cite{Gupta2013}.

Solutions have been fragmented across a variety of application domains such as communication networks \cite{Jiang2006, Szymanski2004, Ye2000, Portnoy2001, Zhen2006, Warrender1999, Angiulli2007, Lane1999, Lane1997, Hofmeyr1998, Sequeira2002}, economics \cite{gupta2012community, Otey2006, Zhu2003}, environmental science \cite{Hill2010, Hill2010a, Angiulli2007, Birant2006, Cheng2006, Yuxiang2005, Wu2010, Drosdowsky1993, Lasaponara2006, Lu2004}, industrial process \cite{Basu2007, Nairac1999, Dasgupta1996, Bu2007}, biology \cite{Keogh2005, Wei2006}, astronomy \cite{Keogh2005, Yankov2008}, and transportation \cite{Li2009,Ge2010}. The fragmentation of application domains led to a variety of problem formulations \cite{Gupta2013}. Furthermore, there is no good understanding of how the solutions in different application domains compare to each other \cite{Cheboli2010}. Therefore, it is difficult to generalize the performance of a solution formulated in one application domain to its performance in another although they might have common elements at a higher level.

Furthmore, adding to the difficulty of comparing solutions, the mechanics of anomaly detection have come from two disciplines: statistics and computer science. Solutions from statistics focus on mathematical rigor while solutions from computer science consider computational issues \cite{Gupta2013}.

\cite{Gupta2013} offers a survey of of anomaly detection problems in a variety of settings such as streaming data, distributed data, and databases. To focus the solution presented here, the problem will be stated as follows: given a finite time series $X$
 \[ X=\{X_1,X_2,X_t,\ldots,X_T\}\]
\[  X \in \mathbb{R}^n \]
where $T$ is the length of the regularly spaced sequence and $n$ is the number of variables $X$ can have, find points in which can be considered anomalous. This statement makes sense only when anomlous points are a small part of the data. Furthermore, anomalies may not even be present.

So elements of any solution to this problem must answer the following questions:

\begin{enumerate}
\item
What is normal (as an anomaly is defined as what is \textit{not} normal)?
\item
What measure is used to indicate how anomalous point(s) are?
\item
How is the measure tested to decide if it is anomalous?
\end{enumerate}



\section[adtypes]{Anomaly Types}

The presense of different anomaly types can be a challenge for anomaly detection techniques. What follow are qualitative descriptions of anomalies classified in a way most relevent to this thesis. However, a taxonomy of anomalies will never encompass all anomalies as well as defining anomalies as points that are not normal.

\subsection[ptanom]{Point Anomalies}

Point anomalies are single points of interest.

\textbf{Simple:} Simple point anomalies are just defined by their value. They are trivial to describe and detect. They are not of much interest in themselves but are mentioned because anomlaies in more complicated time series can be `converted' to resemble this simple type.

\begin{figure}[H]
  \centering
  \includegraphics{figs/trivial.pdf}
  \caption{Simple Point Anomaly}
\end{figure}


\textbf{Context:} Some anomalies are defined within a context. In Figure \ref{fig:contextanom}, the anomalous point's value is within the range of the overall time series. But if the cyclic nature of the series were removed, the anomaly is readily detected (`converted' to a simple point anomaly).

\begin{figure}[H]
  \centering
  \includegraphics{figs/context.pdf}
  \caption{Anomaly in a Periodic Context}
  \label{fig:contextanom}
\end{figure}


\subsection{Discord}

Anomalies over subsequences are called discords \cite{Cheboli2010}. In Figure \ref{fig:perdiscordanom}, about two cycles in a periodic time series are unlike the other cycles. The repeated units do not have to be periodic as in Figure \ref{fig:aperdiscordanom}. 

\begin{figure}[H]
  \centering
  \includegraphics{figs/discord_per.pdf}
  \caption{Discord Anomaly in a Periodic Time Series}
  \label{fig:perdiscordanom}
\end{figure}

\begin{figure}[H]
  \centering
  \includegraphics{figs/discord_aper.pdf}
  \caption{Discord Anomaly in an Aperiodic Time Series}
  \label{fig:aperdiscordanom}
\end{figure}

\subsection{Multivariate}

Multivariate \footnote{In this text, dimensionality of the time series will refer to the length of the time series as opposed to the number of its variables. Time series anomaly detection literature is inconsistent in the terminology used to refer to these two attributes.} time series add another element of complication for detecting anomalies in them and are not a focus of the thesis. \cite{Cheboli2010} classifies multivariate time series according to a combination of periodicity and synchronicity: Variables in a time series can be synchronous and periodic, syncronous and aperiodic, asynchronous and periodic, asyncronous and aperiodic. Therefore, any deviation from these properties are anomalies.




\section[adproc]{Detection Technique}


In this section, a procedure to find anomalies in time series will be outlined to help answer the questions posed in the introduction of this chapter. The focus will be on issues related to time series as opposed to the general anomaly detection problem. However, there will be a focus on the issues related to solving the general problem of finding anomalies in (an arbitrary) time series as opposed to finding anomalies in a particular application domain. Also, computational issues will not be emphasized.

By examining many detection techniques, a general procedure (Section 2.7 in \cite{Cheboli2010}) can be gleaned:

\begin{enumerate}
\item Compute an anomaly score for an observation (a point or a subsequence in a time series). The anomaly score is a deviation from some `normal' value such as a model or similarity to other observations.
\item Aggregate the anomaly scores for many observations.
\item Use the anomaly scores to determine whether an observation can be considered anomalous. For example, an observation could be considered anomalous if its anomaly score exceeds three standard deviations of other anomaly scores.
\end{enumerate}

The procedure is rather straightforward. But, the process of finding what is normal is not. This section is devoted to explaining why finding anomalies in time series is particularly challenging. Addressing the stated problem, given an (single) arbitrary time series, the process of finding normal behaviour involves the following steps:
\begin{enumerate}
\item Extract Samples
\item Transform Samples
\item Apply Detection Technique
\end{enumerate}

\subsection{Sample Extraction}

Many instances of data are needed to inform what is normal. So, subsequences need to be extracted from a (long) sequence.

Samples can be extracted from a sliding a window over the time series. More precisely, beginning at step $t=0$, sliding a window of width $w$ over a time series $X$ of length $T$ one step at a time produces $p=T-w+1$ windows, $\mathcal{X}=\{W_1,W_2,\ldots,W_p\}$\footnote{In literature, this form corresponds to a time series database \cite{Gupta2013}}.

Now, $\mathcal{X}$ contains all possible subsequences of $X$ of length $w$ and the value $p$ is typically not much less than $T$. Having many subsequences helps to localize the anomaly; so the window capturing the anomaly will have a higher anomaly score than adjacent windows.

But sometimes it is not desirable for computational reasons to process all subsequences. $p$ can be reduced by introducing a `hop', $h$, that skips $h$ steps when advancing the window from the previous one ($h=1$ gives all possible subsequences). 

However, by introduing a large enough hop, anomalies could be missed. Consider the sequence \emph{abccabcabc}. The second \emph{c} is an anomaly. Now inspect the windows generated by various values of $h$ in Table \ref{tbl:hop}. The anomalous \emph{c} is captured in a window when $h$ is 1 or 2 but not when $h$ is 3 or 4. As a general rule, when $h=1$, an anomaly would never be missed. But when $h>1$, there is a chance that anomalies would be missed.

\begin{table}[h]
  \centering
  \begin{tabular}{|c|c|}
    \hline
    hop ($h$) & Windows \\
    \hline
    \hline
    1 & \emph{abc},
        \emph{bcc}, 
        \emph{cca}, 
        \emph{cab},
        \emph{abc}, 
        \emph{bca},
        \emph{cab},
        \emph{abc} \\
    \hline
    2 & \emph{abc},
        \emph{cca},
        \emph{abc},
        \emph{cab} \\
    \hline
    3 & \emph{abc}, 
        \emph{cab},
        \emph{cab} \\
    \hline
    4 & \emph{abc}, 
        \emph{abc} \\
    \hline
  \end{tabular}
  \caption{Windows of width 3 for various hop sizes. From \cite{Cheboli2010}}
  \label{tbl:hop}
\end{table}

Another issue that needs to be considered when working with windows is that the window size must be large enough to capture an anomaly. Consider the sequence \emph{aaabbbccccaabbbcccaaabbbccc} where the fourth \emph{c} is anomalous. The window width must be at least 4 to capture the fourth \emph{c}. 

Now given $\mathcal{X}$ for some $w$, the problem may be posed as a multivariate anomaly detection problem. In this setting, assuming that $X$ is univariate, samples of $\mathcal{X}$ correspond to $w$ variables. However, doing so largely ignores the temporal nature of $X$. These issues will be discussed within the forthcoming parts of this chapter.

Finally, subsequent steps taken in the anomaly detection process may put restrictions on $h$ and $w$. For example, if the window size is too large, there may not be enough samples properly apply an anomlay detection technique.

\subsection{Transformation}

Anomalies can be more easily detected if the time series are analyzed in a different representation. Usually these representations are of lower fidelity but capture the essential characterists of the time series in certain cases. As a general example, a real-valued time series could be or should be discretized into a finite set of symbols or numbers to make use of techniques from text processing of bioinformatics. Or, it could be transformed into a different domain such as the frequency domain to make use of techniques from signal processing. As an added benefit, the tranformed time series need less computation given their reduced representation.

More specifically, the Symbolic Aggregate approXimation (SAX) \cite{Lin2007} is an example of time series discretization used to find anomalies in  \cite{Keogh2005}. While the Haar transform represents a transformation to the frequency domain \cite{Bu2007,fu2006finding} for the same purpose.

However, as a transformation only captures the essential characteristics of a time series, more subtle anomalies could be lost in the transformation process. For example, the anomaly in \ref{fig:contextanom} would be difficult to encode in terms of frequency. The anomaly is localized in time while oscillations (representing frequencies) are not. 

Another issue to consider is the similarity of the arrangment of time series `points' (like elements of $\mathcal{X}$) in the transformed space to that of the original space. Some anomaly detection techniques rely a certain distribution of points in a space. Suppose an anomaly detection technique works in $\mathbb{R}^w$ by identifying points that are far away from some normal cluster of points. This arrangement should also be present in the transformed space. Anomaly detection techniques of this type are introduced in the next section. %todo ref next'

However, a recent empirical study \cite{Wang2013} suggests that, in general, there is little to differentiate between numerous time series representations. Only spectral transformations applied to periodic series showed some advantage but only in certain cases.

\subsection[adtechnique]{Detection Technique}

Discussion of detection techniques are discussed as a separate chapter.

\chapter[adtechnique]{Detection Technique}

The application of anomaly detection in a wide variety of application domains have led to the development of numerous detection techniques. Each of these applications defines anomalies in a different way;  some  may only be interested in single anomalous points while others are more concerned about anomalous subsquences. Furthermore, techniques are developed drawing on theory from statistics, machine learning, data mining, information theory, and spectral theory.

A highest-level categorization of these techniques could be as follows. The categories are not exhaustive but capture a wide variety of techniques discussed in literature.

\begin{description}

\item[Segmentation] In a segmentation-based techniques, the time series is first split into homogeneous segments. Then a finite-state automation is trained to learn transition probabilities between segments. So, a segmented anomalous time series should not have high transition probability \cite{Salvador2005,mahoney2005trajectory,Chan2005}.

\item[Information Theory] Information-theoretic techniques quantify a notion of information content such as entropy. So, a point is considered an anomaly if its removal reduces the information content significantly \cite{Muthukrishnan2004,jagadish1999mining}. That is, anomalous points increase disorder, or require more information to be represented in the sequence.

\item[Proximity] Techniques based on proximity map time series onto a space. It is expected that anomalous time series are `different' because they are far from normal ones.

\item[Model] The difference between the (actual) values of a time series and its predicted values from a model indicate how anomoulous they are.
 
\end{description}


Given the variety of techniques applied in different application domains, it is not always possible to use a solution developed for one problem and apply it to another. Finding a general anomaly detection technique is difficult. To the author's knowledge, only one study \cite{Cheboli2010} attempted to compare anomaly detection techniques over a wide variety of data. The study showed, as expected, varying performance of the techniques. Some explanations were given for the varying performance due to a combination of time series characteristics and algorithm settings. These explanations do not help to objectively determine \emph{a priori} what technique to use and how to adjust any parameters it might use.

It would be difficult to use information theoretic techniques because finding an information theoretic measure sensitive enough to detect a few anomalies is challenging \cite{Chandola2009}. Segmentation-based techniques require that a time series to be made of homogenous segments. These conditions are deemed too restrictive to be able to solve the general problem. In addition, both are not well-studied. So, they are not further explored here.

This leaves model-based techniques and proximity-based techniqes as potential solution categories. Both are widely studied. Furthermore, it is possible to make a theoretical comparison between model-based techniques and proximity-based techniques if they are evaluated as, respectively, generative and discriminative models \cite{Ng2006}. Model-based techniques are usually preferred for anomaly detection \cite{Ngkvist2014} assuming enough training data are available.

Obviously a good model is needed as well; recurrent neural networks will be introduced in the next chapter. However, proximity-based solutions are explored in this chapter as a benchmark for comparison as they are well-studied and have had numerous successful applications. Also, the hidden markov model is introduced as an example of model-based solutions in this chapter as another benchmark.


\section[adprox]{Proximity}

As previously mentioned, proximity-based techniques map time series as points of dimention $w$ in some space using some distance measure. The distance measure is used to evaluate how close a test time series is to others; anomalous time series are those that a far (dissimilar) from those considered normal.

This the implies that the time series `points' are arranged in a certain way in the space. In two dimensions, the simplest distribution is portrayed in Figure \ref{fig:simple_dist}. Normal points, $\mathcal{N}_1$, are somewhat clustered. To test whether $p_1$ is an anomaly, it is easy to see, and calculate, that point $p_1$'s nearest neighbor is larger than the nearest neighbor distances of all other points.

\begin{figure}[H]
  \centering
  \includegraphics{figs/simple_dist.pdf}
  \caption{Simple Anomaly Distribution}
  \label{fig:simple_dist}
\end{figure}

Practically, this idealization never occurs. It is not as simple to distinguish $p_1$ and $p_2$ from $\mathcal{N}_1$ and $\mathcal{N}_2$ in the situations depicted in Figure \ref{fig:hard_dist}.

\begin{figure}[H]
  \centering
  \begin{subfigure}[H]{2in}
    \includegraphics{figs/hard1_dist.pdf}
    \caption{}
    \label{fig:hard1_dist}
  \end{subfigure}
  \begin{subfigure}[H]{2in}
    \includegraphics{figs/hard2_dist.pdf}
    \caption{}
    \label{fig:hard2_dist}
  \end{subfigure}
  \caption{Complex Anomaly Distribution}
  \label{fig:hard_dist}
\end{figure}

Having complex distributions is the purview of anomaly detection in general and not a problem particular to time series data. However, issues influenced by the temporal nature of the data will the focus of the next two subsections.

\subsection{Distance Measures}



%%% Local Variables:
%%% mode: latex
%%% TeX-master: "thesis"
%%% End:
\documentclass{standalone}
\usepackage{pgf}
\usepackage{tikz}
\usepackage{amsmath,amssymb}
\newcommand{\mt}[1]{\ensuremath{\mathbf{#1}}}
\newcommand{\vc}[1]{\ensuremath{\boldsymbol{#1}}}



\usetikzlibrary{arrows,automata,matrix}
\usepackage[latin1]{inputenc}
\begin{document}

\begin{tikzpicture}[->,>=stealth'
  ,shorten >=0pt
  ,auto,node distance=2.8cm,
  semithick]
  \tikzstyle{every state}=[]

  \matrix (m) [matrix of nodes
  ,row sep=.25in,column sep=.25in] {
    \node[     ](y) {$\vc{y}$}; \\
    \node[state](L) {$L$}; \\
    \node[state](o) {$\vc{o}$}; \\
    \node[state,dotted](sn) {$\vc{s}_n$}; \\
    \node[state](s2) {$\vc{s}_2$}; \\
    \node[state](s1) {$\vc{s}_1$}; \\
    \node[     ](x) {$\vc{x}$}; \\
  };

  \path
  (x)  edge  node             {$\mt{\st{W}}_{xs_1}$}   (s1)
  (s1) edge [loop right] node {$\mt{\st{W}}_{s_1s_1}$} (s1)
  (s2) edge [loop right] node {$\mt{\st{W}}_{s_2s_2}$} (s2)
  (s1) edge  node {}                                   (s2)
  (s1) edge  node {}                                   (s2)
  (s2) edge  node {}                                   (sn)
  (o)  edge  node {}                                   (L)
  (y)  edge  node {}                                   (L)
  ;
  \path [dotted] 
  (sn) edge  node {$\mt{W}_{s_no}$}                    (o)
  (sn) edge [loop right] node {$\mt{\st{W}}_{s_ns_n}$} (sn)
  ;
  
\end{tikzpicture}

\end{document}
\chapter[]{Anomaly Detection Using Recurrent Neural Networks}


\section{Introduction}

Chapters \ref{ch:ad} and \ref{ch:rnn}, separately, introduced anomaly detection in time series and recurrent neural networks as time series modelers.
%
This chapter outlines a procedure for finding anomalies using RNNs with the goal of mitigating many of the problems associated with anomaly detection that were discussed in \ref{ch:ad}.
%
As much as possible, the same procedure is applied to some time series.


The outlines describes:
%
\begin{description}
%
\item[sampling:] the time series used for training and its associated sampling process
%
\item[recurrent autoencoders:] the specific form of the RNN
%
\item[training:] the training algorithm used on the RNN
%
\item[Bayesian optimization:] the search for optimized parameters for the procedure
%
\end{description}



\section{Sampling}


Some time series with anomalies were chosen or generated to test the anomaly detection process.
%
While there is only one `master' time series to train on (in each test), the RNN needs to see many `samples' from the master time series.
%
The sliding window procedure, described in \ref{sec:adsample}, is used to get samples of some window length.
%
Additionally, since RNNs do not need a fixed window, sliding windows are obtained for more than one window length.
%
Furthermore, the samples are organized into `mini-batches' (of the same window length) to be more compatible with the training procedure.
%
The window length is incremented (size skip) from some minimum until it is not possible to obtain a mini-batch (so the largest window will be close to the length of the time series).


Table \ref{tbl:winspec} specifies the relevent sizes and increments for the sampling process.
%
The values were adjusted manually until a `reasonable' number of samples were found.
%
However, the minimum window length was chosen such that meaningful dynamics were captured (regardless of the scale of the anomaly).

\begin{table}[H]
\centering
\begin{tabular}{|l||c|c|c|c|c||c|}
  \hline
  series & length & min. win. & slide skip & size skip & batch size & $\Rightarrow$ num. samples
  \\ \hline \hline
  sine (gen.) & 1000 & 100 & 10 & 10 & 30 & 96 
  \\ \hline
  ecg \cite{PhysioNet} & 3277 & 300 & 20 & 20 & 30 & 300
  \\ \hline
  spike (gen.) & 800 & 80 & 8 & 8 & 10 & 378
  \\ \hline
  power \cite{Keogh2005} & 7008 & 100 & 100 & 20 & 30 & 121
  \\ \hline
  sleep \cite{this} & 2560 & 300 & 20 & 20 & 30 & 165
  \\ \hline
\end{tabular}
\caption[]{Time series sample specifications} %todo: what case?
\label{tbl:winspec}
\end{table}



   % ts=get_series(id)
   %  tnth=int(.1*len(ts))
   %  kwargs.setdefault('min_winsize',        int(    tnth))
   %  kwargs.setdefault('slide_jump' ,        int(.10*tnth))
   %  kwargs.setdefault('winsize_jump',       int(.10*tnth))
   %  kwargs.setdefault('batch_size',         10           )
    
   %  if 'ecg'==id:
   %      kwargs['min_winsize']=  300
   %      kwargs['slide_jump']=   20
   %      kwargs['winsize_jump']= 20
   %      kwargs['batch_size']=   30

   %  elif 'sleep'==id:
   %      kwargs['min_winsize']=  300
   %      kwargs['slide_jump']=   20
   %      kwargs['winsize_jump']= 20
   %      kwargs['batch_size']=   30

   %  elif 'sin'==id:
   %      kwargs['batch_size']=   30

   %  elif 'twitter'==id:
   %      kwargs['batch_size']=   30
   %      kwargs['min_winsize']=  4000

   %  elif 'power'==id:
   %      kwargs['min_winsize']=  1000
   %      kwargs['slide_jump']=   500
   %      kwargs['winsize_jump']= 200
   %      kwargs['batch_size']=   30









\section{Training}

for detailed overall trning: 'How large should the batch size be for stochastic gradient descent?'


'advances in optimizing recurrent nn' enhanced sgd still competitive or better
'Equilibrated adaptive learning rates for non-convex optimization' -rmsprop is good. sgd good but issue is just how to adjust weights.


%%% Local Variables:
%%% mode: latex
%%% TeX-master: "thesis"
%%% End:

\chapter{Concluding Remarks}
%todo. rename everything to sequence. seq. includes ts

Now that proximity-based and model-based anomaly detection techniques have been introduced in Chapter \ref{ch:ad}, some comparisons can be made with them given the results from the previous chapter.
%
From there, some qualified conclusions can be made.
%
Mind that, from the start, the comparison is made with techniques that do not require labeled data.
%
Also, the comparison is made as general statements of advantages of RNNs over the alternative.

\begin{description}


\item Hidden Markov Models (model).

      Chapter \ref{ch:rnn}, explained how, fundamentally, RNNs store states more efficiently.
      %
      By itself, this does not provide a functional advantage over HMMs, but this requires an HMM for every sequence length, unlike RNNs.
      %
      Furthermore, while HMMs are powerful, RNNs are fundamentally sequence modelers.


\item HOT SAX (proximity).

      The HOT SAX \cite{Keogh2005} technique (and its variants) is considered a proximity-based technique optimized for sequences that is sensitive to window size.
      %
      While the results show in the previous chapter that window size is important, RNNs have the advantage that, the same RNN can be used to find anomalies at different scales.
      %
      In HOT SAX, a comparison is made for almost all pairs of windows for one window size.
      %
      This thorough comparison may be tolerable for short sequences, but a \emph{trained} RNN can analyze a long sequence for anomalies based on a shorter sample.
      %
      Furthermore, the mathematics of RNNs naturally accept multivariate sequences.


\end{description}


Through the previous discussion, the advantage of using autoencoding RNNs as described in this work, in comparison to other techniques, can be summarized in a few questions.
%
A negative response to the following questions for the alternative gives RNNs an advantage.

\begin{itemize}

\item Does other data need to be accessed?%
\footnote{This is related to generative versus discriminative models. Generative models are preferred for anomaly detection.}
(Is a summary of the data stored?)

\item Is it robust against some window length?

\item Is it invariant to translation? (Is it invariant to sliding a window?)

\item Is it fundamentally a sequence modeler?

\item Can it handle multivariate sequences?

\item Can the model prediction be associated with a probability \cite{Graves2013b}?

\item Does it need labeled data? If not, is it robust to anomalous training data?

\item Does it require domain knowledge?

\end{itemize}


Finding a technique with all these advantages is difficult.
%
But, as mentioned in the Introduction chapter, the work in \cite{Malhotra2015} is the closest to the work described here so some comparison is merited.
%
In \cite{Malhotra2015}, the RNN is trained to predict a set length of values ahead for every (varying length) subsequence that starts at the first point%
\footnote{Clarification provided in electronic exchange with author, P. Malhotra.}% obvioiusly you need this pct sign here!
.
%
Although this training setup was used to avoid dealing with windows (as an advantage), the choice of the prediction length remains arbitrary and its effect on finding anomalies at different scales is not studied.
%
In this work, although windows were found to be needed to detect anomalies, the only choice made regarding their length was to set some minimum meaningful length for the training samples%
\footnote{Another way of seeing the difference in the mode of operation between the two RNN setups is by considering their mappings. In \cite{Malhotra2015}, an arbitrary subsequence is mapped to a fixed length sequence while in this work an arbitrary subsequence is mapped to itself.}
(not the scale of the anomaly).
%
In fact, specifying a window size for the prediction errors (as in Section \ref{sec:results}) can be seen as an advantage because it allows detection of anomalies at different scales as a desired choice for the investigator.
%
Furthermore, \cite{Malhotra2015} uses normal data for training thereby not providing evidence that their process can tolerate anomalous data.
%
But in contrast to this work, evidence for anomaly detection in multivariate time series is provided.
%so each pt has all previous predeicted length times
%they used a set prediction output seq. unlike AE.
%they also do not prove that it works by using ALL data
%but they show multivar unlike here
%how far out makes sense?
%diff pred strenghts for a pt and surely rnn can't predict that far ahead

Unfortunately, the power of RNNs comes at high computational expense in the training process.
%
Not only is there a cost in finding RNN parameters ($\vc{\theta}$), but there is also a cost in finding RNN hyper-parameters which can include parameters specifying RNN architecture as well as parameters specifying training algorithm parameters.


Given the results of this work and how it compares to other techniques, it can be concluded that autoencoding RNNs can be used to detect anomalies in arbitrary sequences, provided that an initial training cost can be managed.


\section{Further Work}

The text ends with a list of further work directions with potential to strengthen the case for using autoencoding RNNs in anomaly detection.
%
As the list is mainly concerned with the RNNs, and much progress has been made in RNN research recently, the list is not exhaustive.
%
Furthermore the rapid progress might render items in the list as outdated in the near future.


\begin{description}[style=unboxed]


\item[Better optimize presented work.]

More training epochs and more LSTM layers could have found more optimized parameters.
%
Also, variations in the training data on the length scale of the sequence (trends) should be removed.
%
These optimizations are important to effectively learn normal sequence patterns.


\item[Use autocorrelation to determine a minimum window width.]

In the sampling process, the minimum window length was manually determined such that the length captured meaningful dynamics.
%
This length can be systematically determined by using information from the sequence's autocorrelation.


\item[Accelerate training.]  \hfill %obviously no need for \\ obviously

                 \begin{description}


                 \item[Normalize input.]

                 Although not required, some carefully chosen normalization of data could help.
                 %
                 Another normalization scheme to consider is found in a a recent paper \cite{laurent2015batch} which suggests using normalization based on mini-batch statistics to accelerate RNN training.
                 

                 \item[Find an optimum mini-batch size.]
                 
                 Some redundancy in the mini-batch is desired to make smooth progress in training.
                 %               
                 However, if the mini-batch size is too large (too redundant), a gradient update would involve more computations than necessary.
                          
                 \end{description}

\item[Use dropout to guard against overfitting.]
%
In this work, to guard against overfitting, a corrupting signal is added which depends on the value of the data.
%
In dropout, regardless of the values of the data, a small portion of nodes in a layer can be deactivated allowing other nodes to compensate.
%
Dropout was first applied to non-recurrent neural networks but recent study \cite{Zaremba2014} explains how dropout can be applied to RNNs.


\item[Experiment with different RNN architectures.] \hfill


                 \begin{description}[style=unboxed]%yeaa just this one but not others!!!


                 \item[Experiment with alternatives to the LSTM layer.]

                 Over a basic RNN, the LSTM imposes more computational complexity as well as more storage requirements (for the memory cell).
                 %
                 Gated Recurrent Units (GRU) \cite{Cho2014} are gaining in popularity as a simpler and cheaper alternative to LSTM layers.


                 \item[Experiment with bi-directional RNNs.]

                 Bi-directional RNNs \cite{Schuster1997} incorporate information in the forward as well as reverse direction.
                 %
                 They have been successfully used with LSTM for phoneme classification \cite{Graves2005}.


                 \item[Experiment with more connections between RNN layers.]

                 A better model might be learned if the layers are connected \cite{Hermans2013} (through weights) because it allows for more paths for information to flow through.

                 \end{description}


\item[Incorporate uncertainty in reconstruction error.]

Thee output from a RNN can be interpreted to have an associated uncertainty \cite{Graves2013b}.
%
It follows that it should be possible to get high or low error signals associated with high uncertainty which should affect the interpretation of the existence of an anomaly (see Reconstruction Distribution, Section 13.3, in \cite{Bengio-et-al-2015-Book}).


\item[Objectively compare anomaly detection performance against other techniques
 over a range of data.] 


While certain disciplines might have benchmark datasets to test anomaly detection, measuring the generality of a technique by evaluating its performance over a wide variety of data is not widespread%
\footnote{Perhaps this is due to the difficulty in finding a general technique.}%
.
%
To solve this problem, Yahoo recently offered a benchmark (labeled) dataset \cite{Laptev2015} which includes a variety of synthetic and real time series.


Methods based on non-linear dimensionality reduction might be competitive \cite{Lewandowski2010}.


\item[Find anomalies in multivariate sequences.]

The NASA Shuttle Valve Data \cite{Ferrel2005} is an example which was used in \cite{Jones2014} and the well-known HOT SAX \cite{Keogh2005} technique.


%questions: why pt errs not reveal anoms?? but this does not further my goal
% what relation b/w stats anom and ML

\end{description}




%todo: fix margin on nested descriptions

%%% Local Variables:
%%% mode: latex
%%% TeX-command-extra-options: "-shell-escape"
%%% TeX-master: "thesis"
%%% End:
%%
%%  bibliography

%% list all of the BibTeX files here for the WinEdt project (if applicable)
%GATHER{bibfile.bib}

%% any bibliography style can be used, but IEEEtran.bst is ideally suited to
%% electrical engineering references

%% include the following directives if there are any appendices
\appendix
\appendixeqnumbering
\include{Appendix}

\bibliographystyle{IEEEtran}
\bibliography{IEEEfull,library}

%%
%% curriculum vitae
%%
\cvpage %todo

\noindent Include your \emph{curriculum vitae} here detailing your background,
education, and professional experience.
\end{document}

