

\chapter[Anomaly Detection in Time Series]{Anomaly Detection in Time Series}

\section[Introduction]{Introduction}

The goal of this chapter is to show that the solution to the general problem of anomaly detection in time series is difficult. A typical general framework for anomaly detection in time series is explained with two advanced solutions as examples and their issues. The proposed solution, explained later, fits into the framework providing a basis for comparison.

Describing the variety of solutions puts this difficulty in context. Few publications survery the problem of anomaly detection for time series in particular \cite{Cheboli2010} \cite{Gupta2013}.

Solutions have been fragmented across a variety of application domains such as acoustics, communication networks, economics, environmental science, industrial process, biology, astronomy, and transportation. The fragmentation of application domains led to a variety of problem formulations \cite{Gupta2013}. Furthermore, there is no good understanding of how the solutions in different application domains compare to each other \cite{Cheboli2010}. Therefore, it is difficult to generalize the performance of a solution formulated in one application domain to its performance in another although they might have common elements at a higher level.

Furthmore, adding to the difficulty of comparing solutions, the mechanics of anomaly detection have come from two disciplines: statistics and computer science. Solutions from statistics focus on mathematical rigor while solutions from computer science consider computational issues \cite{Gupta2013}.

To focus the solution presented here, the problem will be stated as follows: given a finite time series $X$
 \[ X=\{X_1,X_2,\ldots,X_T\}\]
\[  X \in \mathbb{R}^n \]
where $T$ is the length of the sequence and $n$ is the number of variables $X$ can have, find points in which can be considered anomalous. This statement makes sense only when anomlous points are a small part of the data.

So elements of any solution to this problem must answer the following questions:

\begin{enumerate}
\item
What is normal (as an anomaly is defined as what is \textit{not} normal)?
\item
What measure is used to indicate how anomalous point(s) are?
\item
How is the measure tested to decide if it is anomalous?
\end{enumerate}

%%% Local Variables:
%%% mode: latex
%%% TeX-master: "thesis"
%%% End:
