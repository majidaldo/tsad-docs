\chapter[]{Anomaly Detection Using Recurrent Neural Networks}


\section{Introduction}

Chapters \ref{ch:ad} and \ref{ch:rnn}, separately, introduced anomaly detection in time series and recurrent neural networks as time series modelers.
%
This chapter outlines a procedure for finding anomalies using RNNs with the goal of mitigating many of the problems associated with anomaly detection that were discussed in \ref{ch:ad}.
%
As much as possible, the same procedure is applied to some time series.


The outlines describes:
%
\begin{description}
%
\item[sampling:] the time series used for training and its associated sampling process
%
\item[recurrent autoencoders:] the specific form of the RNN
%
\item[training:] the training algorithm used on the RNN
%
\item[Bayesian optimization:] the search for optimized parameters for the procedure
%
\end{description}



\section{Sampling}


Some time series with anomalies were chosen or generated to test the anomaly detection process.
%
While there is only one `master' time series to train on (in each test), the RNN needs to see many `samples' from the master time series.
%
The sliding window procedure, described in \ref{sec:adsample}, is used to get samples of some window length.
%
Additionally, since RNNs do not need a fixed window, sliding windows are obtained for more than one window length.
%
Furthermore, the samples are organized into `mini-batches' (of the same window length) to be more compatible with the training procedure.
%
The window length is incremented (size skip) from some minimum until it is not possible to obtain a mini-batch (so the largest window will be close to the length of the time series).


Table \ref{tbl:winspec} specifies the relevent sizes and increments for the sampling process.
%
The values were adjusted manually until a `reasonable' number of samples were found.
%
However, the minimum window length was chosen such that meaningful dynamics were captured (regardless of the scale of the anomaly).

\begin{table}[H]
\centering
\begin{tabular}{|l||c|c|c|c|c||c|}
  \hline
  series & length & min. win. & slide skip & size skip & batch size & $\Rightarrow$ num. samples
  \\ \hline \hline
  sine (gen.) & 1000 & 100 & 10 & 10 & 30 & 96 
  \\ \hline
  ecg \cite{PhysioNet} & 3277 & 300 & 20 & 20 & 30 & 300
  \\ \hline
  spike (gen.) & 800 & 80 & 8 & 8 & 10 & 378
  \\ \hline
  power \cite{Keogh2005} & 7008 & 100 & 100 & 20 & 30 & 121
  \\ \hline
  sleep \cite{this} & 2560 & 300 & 20 & 20 & 30 & 165
  \\ \hline
\end{tabular}
\caption[]{Time series sample specifications} %todo: what case?
\label{tbl:winspec}
\end{table}



   % ts=get_series(id)
   %  tnth=int(.1*len(ts))
   %  kwargs.setdefault('min_winsize',        int(    tnth))
   %  kwargs.setdefault('slide_jump' ,        int(.10*tnth))
   %  kwargs.setdefault('winsize_jump',       int(.10*tnth))
   %  kwargs.setdefault('batch_size',         10           )
    
   %  if 'ecg'==id:
   %      kwargs['min_winsize']=  300
   %      kwargs['slide_jump']=   20
   %      kwargs['winsize_jump']= 20
   %      kwargs['batch_size']=   30

   %  elif 'sleep'==id:
   %      kwargs['min_winsize']=  300
   %      kwargs['slide_jump']=   20
   %      kwargs['winsize_jump']= 20
   %      kwargs['batch_size']=   30

   %  elif 'sin'==id:
   %      kwargs['batch_size']=   30

   %  elif 'twitter'==id:
   %      kwargs['batch_size']=   30
   %      kwargs['min_winsize']=  4000

   %  elif 'power'==id:
   %      kwargs['min_winsize']=  1000
   %      kwargs['slide_jump']=   500
   %      kwargs['winsize_jump']= 200
   %      kwargs['batch_size']=   30









\section{Training}

for detailed overall trning: 'How large should the batch size be for stochastic gradient descent?'


'advances in optimizing recurrent nn' enhanced sgd still competitive or better
'Equilibrated adaptive learning rates for non-convex optimization' -rmsprop is good. sgd good but issue is just how to adjust weights.


%%% Local Variables:
%%% mode: latex
%%% TeX-master: "thesis"
%%% End:
